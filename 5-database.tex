%!TeX root = main.tex
\documentclass[main.tex]{subfiles}

\begin{document}

\chapter{Database approach}
\vspace*{-1\baselineskip}

\section{Screening studies}

Research on the adsorption properties of zeolites in particular, and nanoporous materials in general, aims at understanding the nature and respective importance of the different interactions that drive the adsorption phenomenon. Afterwards, this understanding can be turned into principles, which are then applied to predict the adsorption capacities of similar materials on similar gases, where these principles remain valid.

The natural starting point for this research is a study of a particular material with a particular gas, which leads to particular observations that may then be generalized. Many such studies have been performed and published on zeolites, each involving a considerable amount of work, which provide very precise results on a given topology, with up to a few \SiAl ratios. Yet, the conclusions can seldom be carried over to provide predictions on other topologies.

The converse strategy consists in doing screening studies, in which a vast array of structures is tackled at the same time. Given the cost of doing so experimentally, these studies are mostly computational, thus providing less precise results than actual experiments, but on a broader scale.

Most zeolite screening studies focus on all-silica zeolites, an approach that sidesteps the issue of placing cations. However, most zeolite adsorption applications require cationic zeolites, since cations critically increase the adsorption capacity of the materials.

Even among the few studies which tackle cationic zeolites across topologies and \SiAl ratios, not all properly place cations in the framework, although it is known that this can lead to incorrect subsequent adsorption measurements [REF]. For example, [10.1021/acs.langmuir.2c03089] targets 57 different topologies with 5 different possible cations, but fails to mention where these are placed before the start of the Grand Canonical Monte-Carlo (GCMC) simulations, likely resulting in them being trapped in a non-optimal placement. [10.1038/NMAT3336] also screens cationic zeolites, but places cations at the so-called ``minimum energy positions'' instead of using a robust method like parallel tempering; they also use a force field made for FAU zeolites only, which is inadequate to screen other zeolites.
We are only aware of the works of Sholl [REF], which tackles many zeolite with various \SiAl ratios for screening purposes, that properly places cations through parallel tempering.

In this chapter, we explain how we constitute a database of cationic zeolite structures and how it can be used for isotherm prediction.

\section{Adsorption isotherm}

\subsection{Definition and classification}

\begin{figure}[b]
	\centering
	\includegraphics[width=0.6\columnwidth]{figures/database/IsothermTypes.png}
	\caption{Classification of gas isotherm by IUPAC, taken from \cite{Rouquerol2013}, adapted from \cite{IUPAC1985}. $p^0$ designates the saturation pressure of the adsorbate.}\label{fig:IUPACisotherms}
\end{figure}

The behaviour of a materials with respect to the adsorption of a gas is typically assessed by measuring the evolution of its adsorption capacity with pressure, at a fixed temperature. The resulting curve is called an isotherm, and it is usually classified according to its general shape into one of several possible categories, represented on \cref{fig:IUPACisotherms}.

The shape of the isotherm depends on the surface of the materials, the strength of its interaction of the adsorbate and of the adsorbate-adsorbate interactions. For example, type I isotherms are typical of materials with very small external surface and monodisperse pores of small size, where adsorption quickly reaches saturation. On the other hand, type III isotherms occur when the adsorbent-adsorbate interaction is weak and adsorption is not limited by the pore size (either because adsorption occurs on the external surface or because the pores are very large). Some of these types exhibit hysteresis loops: for example, type IV(a) isotherms, typical of mesoporous adsorbents, have the adsorption branch lower than the desorption branch (when the gas is withdrawn), which comes from the capillary condensation of the adsorbate at high pressure. Real isotherms are often combinations or slight variations of these idealized types. [10.1007/s10450-016-9766-0]

At very low pressure, all isotherms are expected to be linear: this is Henry's law, and Henry's adsorption constant $K$ is proportional to the slope of the isotherm. In that limit, all adsorbates behave like ideal gases since, by definition, there is no adsorbate-adsorbate interaction at low enough pressure.

The other end of the pressure spectrum is more complex. When adsorption occurs on the external surface of the adsorbent, for instance with clays, and if adsorbent-adsorbent interactions are strong enough to allow multi-layer adsorption, then the number of adsorbed species never stops growing with increasing pressures. However, with zeolites and more generally any adsorbent where adsorption occurs in finite pores, the adsorption capacity plateaus at high pressure when all the pores are filled with the adsorbate. It should be mentioned though that the adsorption capacity can still increase beyond that limit, because of the compressibility of the adsorbate in liquid or supercritical state. These regimes of extremely high pressure (beyond $\qty{100}{MPa}$) are of lesser industrial relevance because reaching these conditions is costly and hazardous. [REF]

\subsection{Experimental observation and numerical prediction}

\label{experimentalisotherm}

Experimentally, adsorption isotherms are drawn by progressively increasing the input pressure of the adsorbate in the material at the fixed temperature, and measuring the amount adsorbed. The desorption isotherm follows up by progressively decreasing the input pressure, and similarly measuring the amount of adsorbate.

The measurement itself usually consists in either of the three following methods:
\begin{itemize}
	\item volumetry: a fixed amount of gas at the target pressure is put in contact with the adsorbent, and what is measured is the difference of pressure before and after adsorption.
	\item breakthrough: the gas starts flowing through the adsorbent at a given time, and the outlet concentration and volumetric flowrates are measured.
	\item gravimetry: the difference in mass of the sample containing the adsorbent before and after adsorption is measured by balancing its weight against buoyancy.
\end{itemize}
By reproducing the experiment with a reference gas (usually \ce{N_2} or \ce{Ar}), these techniques give access to the net adsorption, which is the absolute amount of adsorbed species minus that which would be in a fluid occupying the same space as the adsorbent at the same pressure and temperature. Knowledge of the volume of adsorbent allows retrieving the absolute adsorption capacity, which is directly accessible in simulations.

Some more anecdotic isotherm measurement techniques including impedance spectroscopy [REF] as well as NMR [REF] allow obtaining directly the absolute adsorption, but these methods are much too costly to be used routinely.

The adsorption capacity can be predicted numerically through GCMC simulations, as explained in \cref{GCMC}. Beyond the limits due to the precision of the energy computation, either from a force field or from DFT, and convergence of the simulations, those are also dependent on a model of the adsorbent which cannot take into account the complexity of real materials. In the case of zeolites, the location of the aluminium and of the cations has already been discussed through \cref{cationzeolites}, but more generally, all kinds of defects and small crystal size effects can lead to experimental isotherms differing from the simulated counterparts. This limits the accuracy of individual numerical simulation compared to precise experimental adsorbents, but they remain useful to extract tendencies and direct the search for new adsorbents, which will be experimentally tested eventually.

\section{Database constitution}

Finding general tendencies in the adsorption behaviour across zeolites requires doing the analysis of many different structures. To do so, the first step consists in establishing a database of models for the materials themselves. The general simulation workflow, starting from the idealized zeolite structure defined by its topology, down to the cation placement, has already been described in \cref{cationzeolites}. To summarize, the two steps are:
\begin{enumerate}
	\item Aluminium placement. Start by finding the minimum \SiAl ratio by replacing as many silicium with aluminium atoms as possible, with random exchanges and restarts. Check multiple supercell sizes. Then, starting from these filled structures, replace some aluminium by silicium atoms at random and do more exchange steps to generate 6 aluminium placements for each target \SiAl ratio.
	\item Cation placement, using either parallel tempering or shooting star simulations. The only cation used is sodium.
\end{enumerate}

For the entire database, we used the same force field which is that developed by [REF BoulfelfelSholl2021] for adsorption of small gases on cationic zeolites with mobile cations.

\subsection{Aluminium placement}

The current database contains 6 aluminium placements for all 239 known and not interrupted zeolite topologies, with \SiAl ratios of \num1, \num{1.23}, \num{1.4}, \num{1.7}, \num{2}, \num{3}, \num{5} and \num{10}. The actual \SiAl ratios of each structures are those closest to the previous references, which may sometimes differ. When the minimum \SiAl ratio (presented in \cref{table:zeosial}) is not among the previous references, it is added separately, and the lower \SiAl ratios cannot be generated. For example, the list of actual \SiAl ratios for topology YUG is $1.56$, $\frac{121}{71}\approx1.704$, $2$, $3$, $5$ and $\frac{349}{35}\approx9.971$.

The algorithm sometimes fails at finding 6 distinct aluminium placements: in that case, the maximum number of distinct placements is kept. To check whether a new placement is distinct from previous ones, the following algorithm is used. First, for each topology, the list of T-sites is mapped to the list of integers between $1$ and $N$, the number of T-sites of the topology. Each placement is represented by a sequence of $N$ bits, where a bit of 1 at position $i$ indicates an aluminium in T-site $i$ and 0 stands for silicium. Then, all $M$ symmetries of the topology are applied to the placement: this results in $M$ $N$-bit sequences. The (lexicographically) smallest among those serves as a unique signature that identifies the aluminium placement. Then, when a new placement is found, is signature is computed as previously and compared to those already encountered: if it is different from all of them, the new placement is stored.

In the case where $\SiAl = 1$ for example, there can be up to only 2 distinct aluminium placements that obeys L\"owenstein's rule: taking an arbitrary T-atom as a reference, one placement corresponds to that reference set to Al, and the other to that reference set to Si. For many topologies, these two placements are actually symmetry-equivalent, so they share the same signature and only one of them is actually stored. This is not necessarily the case however: for example, topology JSW has two distinct aluminium placements for $\SiAl = 1$.

Overall, \num{8983} different structures with aluminium placed were generated.

\subsection{Cation placement}

For all 239 known and non-interrupted topologies, and for each of the up to 6 different aluminium placements corresponding to the minimum \SiAl, a shooting star simulation was launched with the hot run at \qty{2000}K for \num{20000} cycles (and \num{2000} initialization cycles), spawning one cold simulation every \num{100} cycles. Each of the resulting \num{200} cold simulations run at \qty{300}K for \num{10000} cycles. This served as an experiment to assess the convergence of the simulations: out of the 239 topologies, 57 converged in less than \num{5000} cycles (averaged across aluminium placements). For these topologies, shooting star simulations were run for all previously created aluminium placements across the different \SiAl ratios, using the same parameters (except for the number of cold cycles raised to \num{15000}). For shooting star simulation, the \num{200} cold simulations are divided into six groups, and for each of these groups, the structure with the lowest corresponding energy is kept. Hence, this methodology yields 6 cation placement per structure.

% TODO: discuss and choose whether lowest energy is better than random.

Overall, \num{14268} different structures with both aluminium and cations placed were obtained.
% TODO: update this number


\section{Prediction}

Using the structures from the database, it is possible to study the adsorption behaviour of zeolites at scale, across topologies and \SiAl ratios. To do so, one simply needs to run a GCMC simulation for each combination of zeolite structure (including \SiAl ratio, aluminium placement, nature and placement of the cations), temperature, pressure, and gas.

In order to investigate the relationship between the topology of zeolites and their adsorption properties, we attempted to build a predictive model that could output a plausible isotherm for each combination of three parameters, which are 1) the zeolite topology, 2) the \SiAl ratio and 3) the temperature, using \ce{Na^+} as cation and either \ce{CO_2} or \ce{N_2} as gas.

\subsection{Why isotherms?}

Adsorption isotherms are often used for the characterization of experimental structures. The technique consists in measuring the isotherm using one of the methods presented in \cref{experimentalisotherm}, then fitting the obtained curve against a set of pre-computed isotherms, called the kernels. Each kernel is computed based on a pore geometry (\textit{i.e.} its shape, like a slit or a sphere, and its size) and the thermodynamical properties of the gas, often modeled using simple fluid theory. Fitting the isotherm then corresponds to finding coefficient on each kernel such that the sum of the kernels weighted by the coefficients yields the best approximation of the isotherm. The best fit thus provides the repartition of pore geometries based on their corresponding coefficients.

Another typical context of use for isotherm fitting is for the identification of a material, or the evaluation of its purity or degree of crystallization. To do so, a simple method consists in measuring one or more isotherms and comparing them to those obtained on a reference structure, to check if they match. In general data science, two curves can be compared by computing their root-mean-square deviation, but this does not provide much physical meaning to the comparison, and requires that both isotherms have points measured at the same pressures. More often, the reference isotherm has been fitted against one model, hence the isotherms of the experimental structure can be fitted against the same model, so that the coefficients of the model, which carry physical meaning, can be compared.

In our setting, the goal is quite different since we aim at predicting the adsorption capacity from a theoretical structure, instead of an experimental one. The concept of isotherm is actually not strictly necessary to this goal: one could imagine making an black-box model that directly outputs the adsorption capacity given the three parameters (topology, \SiAl ratio, temperature) as well as the gas pressure. This approach, however, leads to a model that cannot be physically interpreted, and that offers no guarantee on the general shape of the isotherms, with the risk of them being unphysical. On the contrary, making a model that directly predicts the coefficients of a given isotherm model from the three parameters yields a result that already embeds the physics of adsorption through the isotherm mode.

To make such a model, the first step thus consists in fitting the simulated isotherm against a given adsorption model, in order to obtain the coefficients associated with each triplet of parameters. Retrieving the relation between the parameters and the coefficient then becomes a matter of data science.

\subsection{Adsorption models}

The literature is rife with adsorption models, either grounded in theory or empirical ones. The precise choice of a model for an isotherm stems from nature of the adsorption when known, otherwise the shape of the isotherm, the nature of the material or even the scientific community. Since we use isotherm models simply as a mean for extracting a common mathematical descriptor of the isotherms across numerous settings, our only criterion is that the model should be rich enough to accurately represent the isotherms. Beyond that, the simpler the model, \textit{i.e.} the smaller the number of coefficients necessary to represent an isotherm, the better it will be for prediction.

Next are the isotherm models which were investigated. This list is not exhaustive, but it covers a sizeable proportion of the most commonly used models for small gas adsorption in zeolites.

\subsubsection{Langmuir}

The simplest adsorption model corresponds to a physical situation where the adsorbate behaves as an ideal gas, binds with the adsorbent in a reversible reaction of equilibrium constant $\beta$. In that setting, the Langmuir adsorption model states that the fraction of occupied adsorption sites is $\beta P/(1+\beta P)$ where $P$ is the pressure, hence the total adsorption capacity is:
\[n_\text{Langmuir}(P) = \alpha\frac{\beta P}{1+\beta P}\]
where $\alpha$ is the maximum population of the site.

This very simple model is not accurate for zeolites however. One of the reasons is that zeolites contain multiple adsorption sites, both spatially -- there are multiple adsorption reactions happening simultaneously -- and energetically -- they have different values of $\beta$.

\subsubsection{$N$-site Langmuir}

One simple way to circumvent the previous issues consists in simply summing multiple Langmuir adsorption models. This physically assumes some kind of independence between the different adsorption sites, which is debatable, and the general formula is
\[n_{N-\text{site Langmuir}}(P) = \sum_{i=1}^N \alpha_i\frac{\beta_i P}{1+\beta_i P}\]

Such a general model can be made to fit any isotherm, simply by increasing the number of sites $N$ until reaching a high enough complexity of the model. In practice, $N=3$ is good enough to fit many isotherms because it the fit can be optimized across 6 coefficients, although the underlying physical model may not be accurate at all.

\subsubsection{Freundlich}

This simple empirical model is commonly used when the adsorption is known to happen on a heterogeneous surface, and depends on two parameters $A$ and $\gamma$:
\[n_\text{Freundlich}(P) = A\times P^\gamma\]

Beyond its simplicity, the main limit of the model is its behaviour at high temperature: the model diverges to $+\infty$ whereas, in practice, adsorption reaches saturation (before the highly condensed phase).

\subsubsection{Sips}

The Sips model combines both Langmuir and Freundlich isotherms, with the following formula:
\[n_\text{Sips}(P) = \alpha\frac{(\beta P)^\gamma}{1+(\beta P)^\gamma}\]

This model has two limit regimes: at low pressure, it is equivalent to the Freundlich model while at high pressure, it reaches a plateau similarly to the Langmuir model. It also reduces to the Langmuir model in the particular case of $\gamma = 1$.

\subsubsection{Toth}

The Toth model is another empirical variation of the Langmuir model:
\[n_\text{Toth}(P) = \alpha\frac{\beta P}{\paren{1 + (\beta P)^\gamma}^{1/\gamma}}\]

Like the Sips model, it reduces to the Langmuir model for $\gamma = 1$, and reaches a plateau at high pressure. A slight difference with the Sips model is the adsorption capacity at low pressure, which becomes proportional to $P$, which obeys Henry's law, whereas both Freundlich and Sips models have a dependency in $P^\gamma$.

\subsubsection{Redlich-Peterson}

Like the Sips model, the Redlich-Peterson model combines both Langmuir and Freundlich models:
\[n_\text{Redlich-Peterson} = \alpha\frac{P}{1+ (\beta P)^\gamma}\]

It also follows Henry's law by reducing to a linear function of $P$ at low pressure, but its behaviour at high pressure is similar to Freundlich's, making it only adapted to pressure regions below the saturation point.

\subsubsection{Jensen-Seaton}

By adding a fourth parameter, Jensen and Seaton [REF 10.1021/la9509460] propose an adsorption model that present both Henry's law linear behaviour at low pressure, as well as a non-constant affine behaviour at high pressure that models the compressibility region.
\[n_\text{Jensen-Seaton} = \alpha P\times\paren{1 + \paren{\frac{\alpha P}{\delta(1 + \beta P)}}^\gamma}^{-1/\gamma}\]

It has been used to accurately described the isotherm of \ce{CO_2} adsorption in some FAU zeolites, among others.

\subsubsection{Sips-Toth}

A custom-made adsorption model, absent from the literature, consists in using a fourth parameter $\delta$ to act as a continuous switch between the Toth model ($\delta = 0$) and the Sips model ($\delta \to +\infty$):
\[n_\text{Sips-Toth} = \alpha\times\paren{\frac{(\beta P)^\gamma}{1 + (\beta P)^\gamma}}^{\frac{1 + \gamma\delta P}{\gamma(1 + \delta P)}}\]

Like its two building blocks, this model collapses to a simple Langmuir isotherm when $\gamma = 1$, and reaches a plateau at high pressure. It also obeys Henry's law at low pressure.

\subsubsection{Summary}

\Cref{table:adsorptionmodels} summarizes the key properties of the previously presented isotherm models. The number of coefficients characterizes the complexity of the model. Its behaviour at low pressure should obey Henry's law, while it should be affine at high pressure to correctly model the compressibility phase, or at least behave as a plateau otherwise to account for the stationary phase at intermediate pressures.

\begin{table}[h]
	\centering
	\begin{tabular}{|c|c|c|c|}
		\hline
		\bf Name & \bf Coefficients & \bf Henry's law at low $P$ & \bf Behaviour at high $P$\\\hline
		Linear & 1 & \cellcolor{red!25}No & \cellcolor{green!25}Affine\\\hline
		$N$-site Langmuir & $2N$ & \cellcolor{green!25}Yes & \cellcolor{yellow!25}Plateau \\\hline
		Freundlich & 2 & \cellcolor{red!25} No & \cellcolor{red!25}Polynomial \\\hline
		Sips & 3 & \cellcolor{red!25}No & \cellcolor{yellow!25}Plateau \\\hline
		Toth & 3 & \cellcolor{green!25}Yes & \cellcolor{yellow!25}Plateau \\\hline
		Redlich-Peterson & 3 & \cellcolor{green!25}Yes & \cellcolor{red!25}Polynomial \\\hline
		Jensen-Seaton & 4 & \cellcolor{green!25}Yes & \cellcolor{green!25}Affine \\\hline
		Sips-Toth & 4 & \cellcolor{green!25}Yes & \cellcolor{yellow!25}Plateau \\\hline
	\end{tabular}
	\caption{Main characteristics of different adsorption models}\label{table:adsorptionmodels}
\end{table}

Several of these models can be combined by simply adding them together. For this purpose, one extra dummy model was added, called ``Linear'', which simply consists in doing a linear regression of the isotherm:
\[n_\text{Linear} = \zeta\times P\]

Of course, this model does not fit any actual isotherm by itself, but it can be added to any model that reaches a plateau at high pressure to fix its behaviour for the compressibility phase, by making it affine as expected.

\subsection{Isotherm fitting}

In practice, fitting an isotherm means finding the parameters $\alpha$, $\beta$, $\gamma$, \ldots of the model such that the theoretical isotherm resulting from the model is as close as possible to the input one. This closeness can be properly defined through the root-mean-square deviation for instance. Finding the best parameters can thus be formulated as an optimization problem, \textit{i.e.} a problem of the form
\[\text{minimize }f(\alpha, \beta, \gamma)\\\text{such that }\alpha>0, \beta>0, 0<\gamma<2\]
for example, where $f$ represents the deviation described before.

Unfortunately, the expressions of the different adsorption models make this problem non-convex, which is a category of optimization problems generally considered difficult. To solve it, we use the \texttt{BlackBoxOptim.jl} Julia package to do global optimization. Crucially, this package does not rely on $f$ being differentiable, since that would require finding the derivative of the deviation with respect to the parameters for all investigated adsorption models and their combination.

\texttt{BlackBoxOptim.jl} uses meta-heuristic and stochastic algorithms to perform the actual optimization, but it needs a starting point, \textit{i.e.} an initial value for the parameters, to work. The result often depends on the quality of this initial point: the closer it is to the optimum, the better the result.


One approximate solution consists in linearizing the formulas of the adsorption models, solving the optimization problem on this simple formula through linear regression


\subsection{Simple models}


\subsection{Machine learning}

% topology

\subsection{Perspectives}

% Other zeolite materials



\end{document}
