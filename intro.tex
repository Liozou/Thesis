%!TEX root = main.tex

\chapter*{Introduction}

Gas separation is the action of extracting, from a gas mixture, some of the components from the others. This critical industrial process underlies the purification of all gases, as well as the removal of unwanted species from gas mixtures, for instance toxic or polluting components in exhaust fumes. Direct applications include \ce{CO2} capture, either from industrial sources or directly from air, \ce{O2} purification for medical purposes, \ce{H2} production, which is becoming a major stake for the energy transition, cold generation by liquefying pure \ce{N2} or \ce{He}, radioactive rare gas capture from nuclear plant emissions, and many others. All of these highlight the relevance of this topic both for the industry in general, but also particularly in the current context of environmental crisis.

Unfortunately, gases are difficult to handle. They are often invisible, many of them odorless, and they naturally escape into any reachable space if not properly sealed at all times. Some, like \ce{H2} are flammable and explosive when in contact with the air. Others, like \ce{CO} or \ce{Cl2}, are highly toxic. Their containment usually involves storing them at high pressure, with other associated risks. For human life in particular, any perturbation of the local oxygen concentration can be dangerous and even lethal, which makes any gas leakage a threat in closed spaces. In the context of chemistry, their low concentration generally makes them poor reactants under normal conditions, and all the previous properties make their manipulation hazardous and impractical.

Nonetheless, a few techniques have been developed to enable gas separation. One consists in doing cryogenic distillation, in which the mixture is first cooled until it liquefies, then undergoes fractional distillation: since the different constituents have distinct boiling temperatures, each component can be extracted selectively. This process is too costly to be operated industrially, except to handle very large quantities of gas. It is thus mainly used for air separation, to collect \ce{N2}, \ce{O2} and \ce{Ar} typically.

The other main technique is adsorption. This consists in making the mixture flow through a medium, the adsorbent, that selectively traps some of its components while letting the others flow through, where they are collected. In a second stage of the process, the captured gas is also collected and the adsorbent is regenerated, either by increasing the temperature or by decreasing the pressure compared to the first stage, sometimes both. The precise temperatures and pressures used in this process depend on the adsorbent and the nature of the gases to separate.

Other gas separation techniques exist: membranes technology can be used for dehumidification and \ce{H2} extraction, amine scrubbing for \ce{H2S} and \ce{CO2} separation from flue gas, fractional condensation is used for heavier gas resulting from pyrolysis vapours. But they are not as much used at large in the industry, more for specific cases only.

Adsorption carries the unique promise of being theoretically able to separate any gas composition, as long as a good adsorbent can be found. Such adsorbents need to have a large exchange surface in order to quantitatively interact with the gas mixture, therefore good candidates are usually porous materials, with pore sizes around the nanometer scale. If the two gases being separated have very different molecular sizes, then the perfect adsorbent is one in which the smaller gas can be trapped while the biggest one cannot enter the pore. But apart from this case, there is a large variety of possible materials to try.

Experimentally testing a candidate framework material requires synthesizing it, characterizing its mechanical and thermal properties to ensure it can survive industrial use, and of course checking its adsorption properties with respect to the target gas mixture. This entire process is very costly and time-consuming, so it cannot be used as a primary mean to explore all possibilities. Ideally, only a select few frameworks should be experimentally tested, resulting from a pre-selection based on non-experimental criterion.

In this context, the numerical simulation of nanoporous materials offers a desirable venue. Running such simulations is considerably faster and cheaper than experiments, which makes it possible to screen through the diversity of possible frameworks, and select only those that yield the best properties in simulation, to check against experimental testing. It can also provide insights on structure-property relations in materials, which can be useful to guide the exploration of this search space.

Even in the realm of simulation however, each individual computational method corresponds to a unique compromise between accuracy and cost. Finding optimal candidates thus requires a multi-scale approach, where a first screening covers as many materials as possible, with a possibly low accuracy, and whose resulting candidates are filtered by a second, smaller but more precise study, and so on, until yielding candidates of appropriate quality. It is thus crucial that simulation methodologies be available to cover this range of requirements.

\vspace{2em}
%\begin{center}
%	\pgfornament[width=6cm,color=CTsemi]{88}
%\end{center}
%\vspace{1em}
%\clearpage

This thesis focuses on the development of such methodologies for the study of adsorption of small gases in crystalline nanoporous frameworks. Most of my work thus consists in the elaboration and the implementation of new algorithms to solve problems of interest in the field. My aim is to provide ready-to-use tools to tackle these problems, as well as methodological guidelines, for both experimentalists and other theoreticians.

The \hyperref[topology]{first chapter} of this manuscript is dedicated to the CrystalNets.jl software I developed, which is a toolbox for the identification of the topology of crystal materials. This notion of ``topology'', explained therein, underlies the geometric description of all crystalline frameworks, and should thus correlate with many of their properties. CrystalNets.jl was also adapted into a website, making it easily accessible for non-theoreticians.

The \hyperref[cationzeolites]{second chapter} covers the problem of placing cations in zeolites. Zeolites are a particular class of crystalline nanoporous materials, already widely used in the industry for adsorption and many other applications. Numerically finding the positions of their extra-framework cations is a difficult task however, yet it is a necessary step before attempting to predict their adsorption properties. Both existing and new methodologies are discussed, including a new meta-algorithm for accelerating Monte Carlo simulations in sharp potential energy landscapes. This chapter also details a number of useful numerical techniques I used to decrease the computational cost of simulations.

The simulation of the adsorption phenomenon itself is treated in the \hyperref[adsorption]{third chapter}. The most commonly used methodology is compared to another Monte Carlo scheme made faster by restricting movements to a grid, and which shows excellent agreement for monoatomic species. Again, a few numerical optimizations are explained.

Finally, the \hyperref[database]{fourth chapter} tackles the issue of very large-scale screening. Using a database approach, it uncovers a few preliminary results on the search for a universal model able to predict adsorption isotherms of small gases in zeolites across topologies, \SiAl ratios, cations and temperature, at minimal computational cost and quantifiable precision.


\vfill
\begin{center}
	\pgfornament[width=6cm,color=CTsemi]{75}
\end{center}
\vfill\vfill
