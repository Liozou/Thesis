%!TEX root = main.tex

\chapter*{Conclusion and perspectives}

Crystalline nanoporous frameworks constitute a major source of adsorbent candidates for gas separation applications. In order to efficiently screen through them, numerical methods rely on descriptions of the materials to predict their adsorption properties. At the most precise level, this description needs every atom of the framework to run molecular simulations, in which all atomistic interaction pairs are accounted explicitly; at the coarsest level, the bare topology of a material provides a very simplified view on its microstructure.

This notion of topology was made more accessible to the scientific community by the creation of the \texttt{CrystalNets.jl} Julia library. Its intuitive web interface\footnote{accessible at \url{https://progs.coudert.name/topology}}, with a dedicated visualization panel, allows easily identifying the topology of crystalline materials without computational knowledge. I also developed additional topology analysis tools in the publicly released \texttt{PeriodicGraphs.jl} Julia package\footnote{documentation available at \url{https://liozou.github.io/PeriodicGraphs.jl/}}, including an efficient ring statistics analysis component.

In the case of zeolites, an atomistic description of the microstructure of the material requires knowing the repartition of the aluminium atoms among the T-atoms and the placement of the extra-framework cations in the unit cell. These two problems are numerically difficult to solve and little experimental data is available for most zeolite topologies. I have presented a multi-step methodology to identify the different cationic sites, using a new meta-algorithm called ``shooting star'' to easily parallelize computations, combined with annealed site hopping to correctly populate the obtained sites. These methods have been tested using classical force fields, but they do not rely on those and could be used with other energy computation schemes such as density functional theory.

Once the atomistic structure of the adsorbent is known, its adsorption capacity for a target gas can be computed using grand canonical Monte-Carlo (GCMC) simulations. I have implemented a simplified GCMC strategy where the guest molecules have to be localized on a regular grid, to attempt to improve the performance of these simulations. The results perfectly reproduce those obtained with the initial more costly methodology for monoatomic molecules, but for polyatomic molecules the grid GCMC strategy only gives correct results at low pressure.

Another venue to predict adsorption capacities consists in using machine learning. For this purpose, I have used the previous methodologies to build a database containing adsorption results for \ce{CO2} and \ce{N2} in cationic zeolites with \ce{Na+} cations, at different \SiAl ratios, temperatures and pressures. From this, a few simplistic regression models already show promising results for directly predicting adsorption isotherms, hence ensuring that the yielded adsorption capacities still obey an underlying physical model.\\

This work could be extended in several directions. The effect of the nature of the cation on both cationic site location and adsorption across zeolites needs a separate study. For the purpose of gas separation, the prediction of coadsorption warrant a dedicated optimized methodology, although simple models already exist to predict coadsorption from pure components adsorption capacities. Due to its prevalence in many industrial aspects, the competitive adsorption of water in particular should be evaluated for all zeolites of interest. And finally, since gas adsorption is experimentally known to possibly influence the location of the cationic sites in zeolites, this observation needs to be assessed in numerical simulations and its effects carefully evaluated.

For the time being, the next step of the project consists in adding more topologies to the only three (FAU, LTA and CHA) present in the adsorption isotherms database. This would enable the construction of more refined prediction models that could take the topology of the zeolite as a parameter. To do so, different topological descriptors need to be tried to determine which ones correlate best with the computed properties. In this context, the topological tiling of the framework, which consists in its decomposition into volumes limited by the rings of its bond network, could provide key indicators of the pore structure of the zeolite, or even on the cationic sites, which is likely to strongly correlate with the adsorption capacity. Any realistic model will probably also require extra-topological information such as geometric data, to complete its description of the zeolite.

Ultimately, reaching such a model would mean that the adsorption capacity of a gas in a cationic zeolite could be given from a single mathematical formula, for all topologies, \SiAl ratios, temperatures and pressures. The constant values appearing in this formula would stem from computed correlations in simulated data, thus the uncertainty of their values could be obtained, and propagated to yield a controlled uncertainty on the result. In turn, this would pave the way for ultra-fast screening of zeolites for adsorption purposes with controlled accuracy.
