%!TEX root = main.tex
%!TEX spellcheck = fr-FR
\begin{otherlanguage}{french}

% Chapter-style header without adding to the TOC
\hrule\relax
\vspace*{.9\baselineskip}%
{\huge\spacedallcaps{Remerciements}}\par%
\vspace*{1.1\baselineskip}%
\hrule\relax
\vspace*{\baselineskip}%


\begingroup
%\slshape

Cette thèse a beau avoir un unique auteur, elle n'existerait pourtant pas si j'étais seul. Je tiens donc à établir ici quelques remerciements opportuns.\\

Tout d'abord, un immense merci à mon directeur de thèse, François-Xavier Coudert. Merci pour ta direction scientifique bien sûr, qui m'a offert des directions de recherche sans jamais me contraindre, mais aussi pour ta disponibilité et ta réactivité, dont la constance est un vrai confort lors des multiples suprises qui peuvent agrémenter une thèse. Au rang de tes nombreuses qualités humaines, j'ai tout particulièrement apprécié l'atmosphère détendue que tu as réussi à instaurer au laboratoire, oasis dans un monde académique si souvent anxiogène pour les doctorants. Merci pour les conversations du midi sur tous les sujets possibles, des surgelés Picard à l'art du solo de flûte, en passant par l'étymologie japonaise, la place du vélo dans l'urbanisme, la Croix-Rouge, le café, et les jeux de plateau !

Merci à Pluton Pullumbi, mon co-encadrant chez Air Liquide, grâce à qui j'ai pu découvrir le monde industriel, qui m'était alors inconnu. Au travers du leitmotiv de nos rencontres, le classement énergétique des sites d'adsorption, j'ai grandement apprécié ta patience face à ma tendance à explorer plein de sujets avant d'enfin traiter ceux que tu avais à cœur ! Merci pour ta bonne humeur très sympathique, et pour m'avoir instruit sur le fonctionnement de l'entreprise notamment.

Je tiens à remercier bien sûr les autres membres de mon jury de thèse. Caroline Mellot-Draznieks, dont j'avais eu le plaisir de recevoir une excellente direction, mais aussi la joie, la finesse scientifique, et l'énergie au cours de mon stage. Ma rapportrice, Tina Düren, dont j'ai beaucoup apprécié la conversation sur le placement des cations zéolitiques lors de FEZA 2023 en Slovénie, ainsi que les questions fournies de son rapport. Et mon rapporteur, Émeric Bourasseau, pour ta réactivité lors de nos échanges et pour les commentaires détaillés de son rapport. Merci beaucoup pour avoir accepté de constituer mon jury de thèse.

Un grand merci à Anne Boutin pour ton accompagnement sur le projet de placement des cations. Qui eût cru qu'il y ait tant de façon possibles de se tromper en implémentant un algorithme de Monte-Carlo ! Merci pour ton énergie, ton souci du détail et ton goût pour les figures claires et colorées, trois éléments qui ont grandement contribué à mon avancée sur ce sujet. Merci aussi à Antoine Carof et à Guillame Jeanmairet pour votre disponibilité et vos éclairages sur la DFT moléculaire, ainsi qu'à Raphaël Labeyrie pour notre discussion sur le sujet. Merci à Olaf Delgado-Friedrich pour Systre, et à Frank Hoffmann pour sa pédagogie sur la topologie crystalline. Merci enfin aux membres de mes comités de suivi de thèse, Michaël Badawi et Alexis Markovits, pour avoir pris le temps d'échanger lors de nos entrevues annuelles.

Je me dois de conclure ces premiers remerciements académiques en mentionnant les professeurs qui m'ont formé, chacun dans leur discipline et à leur manière, et sans lesquels ce présent manuscrit n'existerait pas. De cette longue liste, je tiens tout particulièrement à remercier M.~Jouan, Mme~Guintrandy, Mme~Rousseau, Mme~Crouzaud, Mme~Maestrati, Mme~Proietti, M.~Touzillier, M.~Olivier et Mme~Zann, dont le précieux enseignement m'a marqué.\\

De ces trois années passées au laboratoire à Chimie ParisTech, je garderai sans doute surtout en mémoire la vie d'équipe, rendue si agréable par ses inoubliables acteurs ! Les autres thésards d'abord, en commençant chronologiquement par Maxime Ducamp dont la légendaire régularité continuera sans aucun doute d'être contée comme un mythe au laboratoire. Nicolas Castel, dont je ne suis toujours pas sûr après deux ans ensemble de savoir interpréter avec exactitude l'ambigu sourire : un grand merci pour tes descriptions éclairantes du fonctionnement de certaines institutions publiques, et pour Gruissan bien sûr, dont je garde chaud dans le cœur le souvenir de la neige, le vélo, et la féérie des eaux\ldots Emmanuel Ren, pour ta franchise, pour tes talents culinaires, pour m'avoir permis d'avoir de passionnantes discussions algorithmiques sur le découpage énergétique d'une maille élémentaire, et pour ta joie si communicative. Merci à vous deux pour le ski également, c'était super chouette ! Enfin, Dune André évidemment, pour ta vision des choses qui ne cesse de m'étonner, pour ton oreille attentive qui me donne toujours envie de bavarder, associée à ta mémoire phénoménale qui me fait dire que je ne devrais quand même pas trop parler, et pour ton souci de vraiment comprendre les choses.

Merci aux post-doctorants dont j'ai eu le plaisir de partager la salle de laboratoire. Ambroise de Izarra, pour ton caractère en or, je n'aurais pas pu rêver d'un voisin de bureau plus sympathique. Merci pour les conversations techniques sur l'intrusion d'eau dans les zéolites, et sur les conversations moins techniques sur les détails sordides de telle université ou de tel musée d'histoire naturelle, on s'amuse toujours bien avec toi ! Luca Brugnoli, dont la discrétion générale n'a d'égale que la subite volubilité lorsqu'un sujet qui te passionne te saisit : merci pour ton appétit scientifique, et merci de nous avoir fait parler anglais à table aussi ! Et Arthur Hardiagon, avec qui j'ai sans doute le plus parlé de MOFs -- un titre de gloire dans ce laboratoire -- pour ta vision du monde, pour les discussions techniques, et pour tes éclairages sur le Brésil, ainsi que l'enseignement.

Merci aux deux stagiaires enfin : Léna, pour nous avoir tenu en haleine lors de tes passages quant à la tenue de ton stage à New-York, et qui revient finalement en thèse ici, et Ayoub, pour les longues discussions à table si fluides et, ne l'oublions pas, tes délicieuses patisseries.\\

Même si je n'y allais qu'une ou deux fois par mois, merci à tous les ingénieurs d'Air Liquide, avec qui j'ai eu un grand plaisir à échanger sur toute sorte de sujet. Un merci un peu particulier à Vincent Ren, grâce à qui j'ai obtenu un canal priviligé pour découvrir l'équipe ; quand même, le hasard fait bien les choses ! Merci aussi à Sébastien Léonard, pour ta tolérance à mon prosélytisme engagé sur Julia, et pour ton humour. Plus généralement, merci à toute l'équipe Physique et Simulation : Pierre Carrère pour ton accueil très sympathique, Iana Sudreau pour ta pédagogie et ton énergie, Guillaume Lodier pour tes questions fines lors des présentations, Benjamin le Creurer pour tes éclairages sur les rouages de l'entreprise, Soufiane Cherroud (attention à ne pas trop apprendre l'humour de Sébastien), Francis Méziat et Bernard Labégorre. Merci à Guillaume Mougin bien sûr, pour ton humanité, ton calme, et ton expertise sur tous les sujets de l'équipe.

Merci aussi aux autres équipes d'Air Liquide : sans les nommer individuellement, à tous ceux d'ex-Computational and Data Science en premier, avec qui j'ai passé quasiment trois ans, et un mémorable teambuilding ! Merci à Joachim Rasser de m'avoir accueilli dans ce grand groupe, et merci à Gaëlle Pommeray pour sa patiente aide administrative, absolument indispensable dans ce cadre. Merci aussi aux collègues de Techniques de Production Primaire que mon équipe a rejoint, et en particulier à l'équipe Adsorption pour nos échanges.\\

Ceux qui me connaissent savent sans doute que j'aime chanter. J'aimerais donc remercier les personnes qui m'ont permis cette respiration indispensable chaque semaine de ma thèse : tout d'abord Jérôme Hénin, sans qui je ne me serais peut-être jamais remis à la chorale, et dont la direction sans faille du chœur de l'IBPC (IBPChante, puis les Atomes) m'a fait voyager à travers de très nombreuses contrées musicales. Merci à Johan Fargeot de m'avoir accueilli au chœur de PSL, qui m'a offert parmi les plus beaux souvenirs de ces dernières années, des concerts aux Invalides ou au Théâtre des Champs-Élysées, aux tournées à Londres ou à Saint-Germain-du-Plain. Merci aussi à Julien Rezak pour ta profonde pédagogie et ton humour décalé, je te dois le plus gros de mes progrès en chant ces dernières années.

À travers ces chœurs, j'ai rencontré de nombreuses personnes extraordinaires qui ont, elles aussi, contribué à ma thèse par le bonheur que leur compagnie me procure : par ordre désalphabétique des prénoms (pourquoi pas ?), Véronique, Valentin, Pierre, Margaux, Hélène, Gabriel, Etty, Clotilde, Clément, Brigitte, Arthur, Anya, et tous les autres que je ne nomme pas mais que je n'oublie pas pour autant. Merci à vous, du fond du c(h)œur !

Ceux qui me connaissent savent sans doute aussi que j'aime le langage de programmation Julia. Je remercie donc ses concepteurs et développeurs, et en particulier Jeff Bezanson, Keno Fischer, Jameson Nash, Viral Shah et Valentin Churavy avec qui j'avais passé un excellent stage à Boston, ainsi que mon encadrant de l'époque, Jan Vitek, pour m'y avoir invité et pour sa grande sympathie. Assurément, ma thèse aurait un autre visage sans vous.\\

La vie sans thèse, sans chant et sans Julia pourrait sembler triste, mais elle ne l'est pourtant point grâce à mes amis qui la rendent si douce, et que je tiens aussi bien évidemment à remercier ! Merci à Wilson, Soline, Randy, Qiaosheng, Pierre, Natasha, Marie-Jeanne, Hugo, Gabrielle, Eliza, Élise, Claire et Antoine du lycée ; Slimane, Matthieu, Côme, de prépa ; Val, Suzon, Grodrigue, Ludi, Zuki, Alice du BDS ; Yann, Julia, Clara de Cambridge ; Mathieu, Marc, Lucas, Luc et tous les autres Info16; Théophile, Miguel, Clara, Antoine et tous les autres Chimie18 ; toute la promo si soudée des Chimie19 ; Rudy, Pauline, Maverick, Maud, Mathias, inclassables dans les autres catégories ; et j'en ai oublié plein. Vous ne semblez former qu'une longue liste ici, mais je pense à chacun d'entre vous individuellement et vous remercie sincèrement.

Il reste quelques amis dont la présence à mes côtés durant ma thèse mérite des remerciements approfondis : mes colocataires. Florentin, merci pour tes efforts de longue haleine pour tenter de me comprendre ; ta tendance croissante à réussir relève de l'exploit dont la gloire te semble, je suis sûr, très douteuse. Merci ton appréciation du malaise, pour les discussions profondes, pour quelques fous rires mémorables que je n'ai qu'avec toi, pour les matrices de covariance, pour le tian, pour le jazz, pour les pâtes très cuites, et pour le cresson à Frangy. Paul, oh là là, que d'années splendides nous avons passées ! Merci pour ta personnalité rayonnante de douceur, pour les gribouillages sur le cours de chimie, pour le froude, pour les Scuds, pour les échanges sur les colles, pour la peur du manque, pour la danse, pour les pâtes \textit{al dente} au pesto, et pour le beurre à Carnac. Martin, merci pour les descriptions désopilantes de ton travail, l'écologie, les discussions informatiques, le cinéma, les patisseries véganes et les dons du sang qui finissent mal. Et Dune, en plus de ce que j'ai déjà pu dire, merci pour les courses de couloir, le tricot, le temps pour lire une BD, la peinture, la levure alimentaire, la plante et les noyaux d'avocat.

Les avants-derniers remerciements sont les plus ingrats : ils n'ont ni droit à la préséance des précédents, ni à la résonnance de la dernière place. Je les réserve donc avec malice à la personne qui lit ces mots ; après tout, c'est pour vous qu'ils sont écrits ! Si vous lisez le reste de cette thèse, c'est que vous aimez les sensations fortes : bravo pour votre courage, et n'hésitez pas à me contacter au besoin.

Cette litanie ne demande qu'à être conclue mais certaines personnes manquent encore à l'appel, évidemment. Merci Papa et Maman, merci Anaëlle et Nicolas, merci Mamie et Alice, merci à toute la famille, et merci aussi à ceux qui ne sont plus là. Je ne m'épanche pas plus sinon je devrais encore remplir des pages, mais je vous embrasse tous bien fort.


\endgroup
\newpage
\cleardoublepage

%\ifweb
%
%\mbox{}\vfill
%\thispagestyle{empty}
%
%Copyright © 2019 Guillaume Fraux
%
%This document is distributed under a Creative Common license CC-BY-SA-NC 4.0
%(Creative Commons Attribution-NonCommercial-ShareAlike 4.0 International).
%
%See \url{https://creativecommons.org/licenses/by-nc-sa/4.0/legalcode} for the
%full text of the license.
%
%\begin{center}
%    \includegraphics[width=15em]{figures/images/by-nc-sa-eu.png}
%\end{center}
%
%\clearpage
%\mbox{}
%\thispagestyle{empty}
%\clearpage
%
%\fi

\end{otherlanguage}
