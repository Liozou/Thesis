%!TEX root = main.tex
% Lionel Zoubritzky PhD thesis -- (C) 2024

% information for cover pages

\institute{Chimie ParisTech}
\doctoralschool{Chimie Physique et\\ Chimie Analytique de\\ Paris Centre}{388}
\specialty{Chimie Physique}
\date{15/10/2024}

\jurymember{1}{Emeric \textsc{Bourasseau}}{CEA Cadarache}{Rapporteur}
\jurymember{2}{François-Xavier \textsc{Coudert}}{CNRS, Chimie ParisTech -- PSL}{Directeur de thèse}
\jurymember{3}{Tina \textsc{D\"uren}}{University of Bath}{Rapporteuse}
\jurymember{4}{Caroline \textsc{Mellot-Draznieks}}{CNRS, Collège de France}{}
\jurymember{5}{Pluton \textsc{Pullumbi}}{Air Liquide}{}

\frabstract{

La séparation et purification des gaz est un enjeu industriel majeur, qui nécessite l'emploi d'adsorbants spécifiques à certaines espèces moléculaires. Afin d'accélérer le développements de nouveaux adsorbants d'intérêt, il peut être utile d'avoir recours à des méthodes de simulations numériques permettant d'identifier des candidats potentiels.

Ces méthodes restent cependant coûteuses en temps et en énergie. Pour tenter d'améliorer leurs performances, cette thèse propose et explore plusieurs avancées méthodologiques sur les différentes étapes de la simulation. En particulier, de nouveaux algorithmes et outils sont présentés pour l'identification de la topologie des matériaux cristallins, pour localiser les cations dans les charpentes zéolitiques, et pour déterminer la capacité d'adsorption d'un gaz dans un matériau poreux. Les résultats de simulation alimentent une base de donnée qui est aussi exploitée pour proposer une prédiction approximative mais ultra-rapide d'isothermes d'adsorption, ouvrant la voie à des stratégies de criblage de matériaux nanoporeux à haute performance.
}

\enabstract{

\todo{Abstract EN}

Gas separation and purification is a crucial industrial issue, which is technologically addressed by fine-tuning the gas specificity of adsorbents. However, the experimental development of new adsorbents is a costly process, hence the need for preliminary computer simulations to test only the most promising candidates.

These simulation techniques remain costly in themselves. In order to improve their performance, this thesis proposes and explores new methodological options across the different steps of the simulation. In particular, new algorithms and tools are presented to tackle the identification of the topology of crystalline materials,  the location of cations in zeolitic frameworks, and the adsorption capacity of a gas in porous materials. The simulation results are then used to feed a database, which allows using a statistical approach to provide coarse but ultra-fast predictions of adsorption isotherms, paving the way for high-performance screening stratgies of nanoporous materials.
}

\frkeywords{simulation moléculaire, matériaux nanoporeux, adsorption, topologie cristalline, zéolites}

\enkeywords{molecular simulation, nanoporous materials, adsorption, crystal topology, zeolites}
