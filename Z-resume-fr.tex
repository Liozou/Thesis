%!TeX root = main.tex
%!TeX spellcheck = fr-FR
% Lionel Zoubritzky PhD thesis -- (C) 2024

\documentclass[thesis]{subfiles}

\begin{document}

\begin{otherlanguage}{french}

\renewcommand{\thesection}{\arabic{section}}
\renewcommand{\thesubsection}{\arabic{section}.\arabic{subsection}}
\renewcommand{\thefigure}{R\arabic{figure}}
\setcounter{figure}{0}
\titlecontents{section}[4.8em]{\addvspace{0.1em}}{\contentslabel{2.2em}}{}{\titlerule*[1pc]{.}\contentspage}[]

\chapter*{Résumé en français}
\startcontents[chapters]
\printpartialtoc

\section*{Introduction}

% \begin{figure}[h]
% \begin{minipage}[t]{.43\textwidth}
% \centering
%   \includegraphics[width=\linewidth]{figures/2-thermo/Enthalpy_0_log.jpg}
%   \caption{\small{\ Pour 8\,401 MOFs avec une sélectivité Xe/Kr favorable ($s\e{0} > 1$), pair-plots entre les différentes grandeurs $\log(s\e{0})$, $\Delta\e{exc}H\e{0}$, $\Delta\e{ads}H\ex{Xe}\e{0}$ et $\Delta\e{ads}H\ex{Kr}\e{0}$ (les enthalpies sont en \si{\kilo\joule\per\mol}) en dehors de la diagonale et la distribution de chaque grandeur sur la diagonale.}}\label{fgr:histo_H_resume}
% \end{minipage}
% \hfill
% \begin{minipage}[t]{.5\textwidth}
% \centering
%   \includegraphics[width=\linewidth]{figures/2-thermo/s_0_vs_s_2080_overview_log.jpg}
%   \caption{\small{\ Sélectivité à 1\,atm de pression en fonction de la sélectivité à basse pression pour une composition 20\pp{}80 Xe/Kr. Les points sont étiquetés selon la différence relative entre les deux sélectivités. Les points violets ont une grande différence relative entre les sélectivités. }}
%   \label{fgr:overview_resume}
% \end{minipage}
% \end{figure}


% \clearpage
% \section*{Conclusion}

% Texte \autocite{citation}

% \vfill
% \begin{center}
%     \pgfornament[width=6cm,color=CTsemi]{75}
% \end{center}
% \vfill\vfill

\end{otherlanguage}

\OnlyInSubfile{\printglobalbibliography}

\end{document}
